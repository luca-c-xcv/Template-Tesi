\chapter{Hello World 1}
\label{ch:capitolo1}

% --- Inizio del Capitolo 1 ---

Capitolo 1...


    \textbf{Hello World!} Today I am learning \LaTeX. %notice how the command will end at the first non-alphabet charecter such as the . after \LaTeX
     \LaTeX{} is a great program for writing math. I can write in line math such as $a^2+b^2=c^2$ %$ tells LaTexX to compile as math
     . I can also give equations their own space:
    \begin{equation} % Creates an equation environment and is compiled as math
    \gamma^2+\theta^2=\omega^2
    \end{equation}
    If I do not leave any blank lines \LaTeX{} will continue  this text without making it into a new paragraph.  Notice how there was no indentation in the text after equation (1).
    Also notice how even though I hit enter after that sentence and here $\downarrow$
     \LaTeX{} formats the sentence without any break.  Also   look  how      it   doesn't     matter          how    many  spaces     I put     between       my    words.

    For a new paragraph I can leave a blank space in my code.

Esempio di una nota\footnote{CleanCode} con citazione a sezione~\ref{sec:sezione1}.

\begin{figure}[H]
    \centering
    \includegraphics[width=0.2\textwidth]{assets/images/logo-dip_blu_hr.png}
    \caption{Esempio di un'immagine}
    \label{fig:immagine1}
\end{figure}

% \begin{lstlisting}[caption={Esempio codice JavaScript},label={Esempio codice JavaScript}, language=JavaScript]
% function foo(num) {
%     const bar = 2;
%     return num + bar;
% }
%
% const result = foo(2);
% // result -> 4
% \end{lstlisting}

% \lstinputlisting[caption={Esempio codice JavaScript},label={Esempio codice JavaScript}, language=JavaScript]{assets/src/test.js}

\AtEndEnvironment{listing}{\vspace{-9pt}}
\inputmintedtesi{js}{assets/src/test.js}{Codice JavaScript}


\section{S1 - Hello World \label{sec:sezione1}}


Sezione 1... con citazione bibliografica~\cite{CleanCode}

\subsection{Sottosezione 1}
\label{subsec:sottosezione1}

Sottosezione 1...

$$\bar{X}\pm\frac{\sqrt{\hat{p}(1-\hat{p})}}{\sqrt{n}}q_{1-\frac{\alpha}{2}} = \hat{p}\pm\frac{\sqrt{\hat{p}(1-\hat{p})}}{\sqrt{n}}q_{1-\frac{\alpha}{2}}$$
