% DOCUMENTS TYPE %
% Carattere dimensione 12
% \documentclass[12pt]{report}
% Per la stampa fronte-retro sostituire con:
\documentclass[12pt, twoside]{report}


% DEFINE PACKAGES %

% Margini (4cm a sx, 2.5cm a dx, 2.5cm in alto, 2.5cm in basso)
% \usepackage[top=2.5cm, bottom=2.5cm, left=4cm, right=2.5cm, centering]{geometry}
% Per la stampa fronte-retro sostituire con:
\usepackage[top=2.5cm, bottom=2.5cm, inner=4cm, outer=4cm, right=2.5cm, centering]{geometry}

\usepackage[italian]{babel} % applicazione regole di scrittura per la lingua italiana

\usepackage[utf8]{inputenc} % codifica UTF-8

\usepackage{scrlayer-scrpage} % stili pagina per il frontespizio
\usepackage[T1]{fontenc}

% \usepackage{mathptmx} % font Times New Roman (simile)

\usepackage{mathpazo} % font Palatino (simile)

% \usepackage[math]{iwona}

% \usepackage[sfmath]{kpfonts} %% sfmath option only to make math in sans serif. Probablye only for use when base font is sans serif.
% \renewcommand*\familydefault{\sfdefault} %% Only if the base font of the document is to be sans serif

\usepackage{graphicx} % inserimento di immagini

\usepackage{csquotes} % per le citazioni "in blocco"
% \usepackage[backend=biber, sorting=nty, ]{biblatex} % bibliografia con pacchetto biblatex (https://ctan.org/pkg/biblatex?lang=en)

\usepackage{titlesec} % per la formattazione dei titoli delle sezioni, capitoli etc.

\usepackage{float} % per il posizionamento delle immagini

\usepackage{listings} % per il codice di programmazione
% Fonte https://en.wikibooks.org/wiki/LaTeX/
\usepackage[chapter]{minted}

\usepackage{amsmath} %pacchetto per matematica
\usepackage{amsthm}	%pacchetto per teoremi




% SET PACKAGES %

\linespread{1.5} % Interlinea

\ifoot[]{}
\cfoot[]{}
\ofoot[\pagemark]{\pagemark}

\pagestyle{scrplain}

\appto{\bibsetup}{\raggedright}

% Formato delle intestazioni
\titleformat{\chapter}[block]
  {\normalfont\LARGE\bfseries}{\thechapter.}{0.5em}{\LARGE}
\titlespacing*{\chapter}{0pt}{-20pt}{25pt}

\renewcommand\listoflistingscaption{Elenco dei codici sorgente} %rename "List of listings"
\newtheorem{thm}{Teorema}
