\usepackage{xcolor}  % stile del codice

\renewcommand{\lstlistingname}{Code}% Listing -> Codice
\renewcommand{\listingscaption}{Code}% Listing -> Codice

% \newcommand{\listingsttfamily}{\fontfamily{sourcecodepro}\small}
\newcommand{\listingsttfamily}{\fontfamily{nimbusmononarrow}\small}

%colors definition
\definecolor{myblue}{HTML}{005CFF}
\definecolor{mygreen}{rgb}{0,0.6,0}
\definecolor{mygray}{rgb}{0.7,0.7,0.7}
\definecolor{mymauve}{rgb}{0.58,0,0.82}
\definecolor{darkgray}{rgb}{.4,.4,.4}
\definecolor{navy}{HTML}{000080}
\definecolor{purple}{rgb}{0.65, 0.12, 0.82}
\definecolor{codepurple}{rgb}{0.58,0,0.82}
\definecolor{backcolor}{HTML}{f3f6f9}
% \definecolor{backcolour}{HTML}{f8f9fa}
% \definecolor{backcolour}{rgb}{0.95,0.95,0.92}
\definecolor{border}{HTML}{eaecf0}



% Stili configurabili del codice (lslisting)
\lstset{
  belowcaptionskip=0.5em,
  backgroundcolor=\color{backcolor}, % choose the background color; you must add \usepackage{color} or \usepackage{xcolor}
  basicstyle=\footnotesize, % the size of the fonts that are used for the code
  breakatwhitespace=false, % sets if automatic breaks should only happen at whitespace
  breaklines=true, % sets automatic line breaking
  captionpos=b, % sets the caption-position to bottom
  commentstyle=\color{mygreen}, % comment style
  deletekeywords={...}, % if you want to delete keywords from the given language
  escapeinside={\%*}{*)}, % if you want to add LaTeX within your code
  extendedchars=true, % lets you use non-ASCII characters; for 8-bits encodings only, does not work with UTF-8
  frame=single, % adds a frame around the code
  keepspaces=true, % keeps spaces in text, useful for keeping indentation of code (possibly needs columns=flexible)
  keywordstyle=\color{codepurple}, % keyword style
  % language=Octave, % the language of the code
  morekeywords={*,...}, % if you want to add more keywords to the set
  numbers=left, % where to put the line-numbers; possible values are (none, left, right)
  numbersep=5pt, % how far the line-numbers are from the code
  numberstyle=\tiny\color{mygray}, % the style that is used for the line-numbers
  rulecolor=\color{border}, % if not set, the frame-color may be changed on line-breaks within not-black text (e.g. comments (green here))
  showspaces=false, % show spaces everywhere adding particular underscores; it overrides 'showstringspaces'
  showstringspaces=false, % underline spaces within strings only
  showtabs=false, % show tabs within strings adding particular underscores
  stepnumber=1, % the step between two line-numbers. If it's 1, each line will be numbered
  stringstyle=\color{mymauve}, % string literal style
  tabsize=2, % sets default tabsize to 2 spaces
  title=\lstname % show the filename of files included with \lstinputlisting; also try caption instead of title
}


% Languages
\lstdefinelanguage{JavaScript}
{
  keywords={typeof, new, true, false, catch, function, return, null, catch, switch, var, const, let, if, in, while, do, else, case, break},
  keywordstyle=\color{myblue}\bfseries,
  ndkeywords={class, export, boolean, throw, implements, import, this},
  ndkeywordstyle=\color{darkgray}\bfseries,
  identifierstyle=\color{black},
  sensitive=false,
  comment=[l]{//},
  morecomment=[s]{/*}{*/},
  commentstyle=\color{mygreen}\ttfamily,
  stringstyle=\color{red}\ttfamily,
  morestring=[b]',
  morestring=[b]"
}




\lstset
{
   language=JavaScript,
   extendedchars=true,
   basicstyle=\footnotesize\listingsttfamily,
   showstringspaces=false,
   showspaces=false,
   numbers=left,
   numberstyle=\footnotesize,
   numbersep=9pt,
   tabsize=2,
   breaklines=true,
   showtabs=false,
   captionpos=b
}

% MY CUSTOMIZATION %

% it takes 3 args: language, file path, caption
\renewcommand\theFancyVerbLine{\footnotesize\arabic{FancyVerbLine}}
\newcommand{\inputmintedtesi}[3]{
    \begin{listing}[H]
      \inputminted[frame=single, %frame around code
                   framerule=0.0pt, %frame size
                   rulecolor=mygray, %frame color
                   framesep=2mm, %frame distance
                   baselinestretch=1.3, %code line hight
                   style=emacs, %style code
                   bgcolor=backcolor, %background color
                   fontfamily=courier, %code font family
                   fontsize=\footnotesize, %code font size
                   tabsize=2, %tab size
                   linenos %view line numbers
                  ]{#1}{#2}
       \label{lst:#3}
       \caption{#3}
    \end{listing}
   }
